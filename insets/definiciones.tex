% Definiciones
\theoremstyle{plain}
%\newtheorem{thm}{Theorem}[chapter] % reset theorem numbering for each chapter

%\theoremstyle{definition}
%\newtheorem{defn}[thm]{Definition} % definition numbers are dependent on theorem numbers
%\newtheorem{exmp}[thm]{Example} % same for example numbers

% \newcommand{\chaptercontent}{
% \section{Basics}
% \begin{defn}Here is a new definition.\end{defn}
% \begin{thm}Here is a new theorem.\end{thm}
% \begin{thm}Here is a new theorem.\end{thm}
% \begin{exmp}Here is a good example.\end{exmp}
% \subsection{Some tips}
% \begin{defn}Here is a new definition.\end{defn}
% \section{Advanced stuff}
% \begin{defn}Here is a new definition.\end{defn}
% \subsection{Warnings}
% \begin{defn}Here is a new definition.\end{defn}
% }


\newcommand\DrawControl[3]{
  node[#2,circle,fill=#2,inner sep=2pt,label={above:$#1$},label={[black]below:{\footnotesize#3}}] at #1 {}
}
\lstdefinestyle{customc}{
   belowcaptionskip=1\baselineskip,
   tabsize=2,
   breaklines=true,
  %frame=L,
  numbers=left,
  xleftmargin=\parindent,
  language=C,
  showstringspaces=false,
  basicstyle=\footnotesize\ttfamily,
  keywordstyle=\bfseries\color{green!40!black},
  commentstyle=\itshape\color{purple!40!black},
  identifierstyle=\color{blue},
  stringstyle=\color{orange},
}

\def\checkmark{\tikz\fill[scale=0.5](0,.35) -- (.25,0) -- (1,.7) -- (.25,.15) -- cycle;}


\newcommand{\xmark}{\text{\ding{55}}}

\hypersetup{pageanchor=false}
\hypersetup{pageanchor=true}

\renewenvironment{leftbar}[1][\hsize]%
{%
        \def\FrameCommand%
        {%
             \includegraphics[width=1cm]{assets/images/warning.png}%           
             \fboxsep=\FrameSep\colorbox{cyan!5}%
        }%
        \MakeFramed{\hsize#1\advance\hsize-\width\FrameRestore}%
}%
{\endMakeFramed}


\newcommand\dir{assets/images/capitulos}
\newcommand\nchapter{0}


% evitamos que las páginas en blanco tengan cabeceras
\makeatletter
\def\clearpage{%
  \ifvmode
    \ifnum \@dbltopnum =\m@ne
      \ifdim \pagetotal <\topskip
        \hbox{}
      \fi
    \fi
  \fi
  \newpage
  \thispagestyle{empty}
  \write\m@ne{}
  \vbox{}
  \penalty -\@Mi
}
\makeatother




\newcommand\abstractname{Abstract}  %%% here
\makeatletter
\if@titlepage
  \newenvironment{abstract}{%
      \titlepage
      \null\vfil
      \@beginparpenalty\@lowpenalty
      \begin{center}%
        \bfseries \abstractname
        \@endparpenalty\@M
      \end{center}}%
     {\par\vfil\null\endtitlepage}
\else
  \newenvironment{abstract}{%
      \if@twocolumn
        \section*{\abstractname}%
      \else
        \small
        \begin{center}%
          {\bfseries \abstractname\vspace{-.5em}\vspace{\z@}}%
        \end{center}%
        \quotation
      \fi}
      {\if@twocolumn\else\endquotation\fi}
\fi
\makeatother





\hypersetup{
	pdfauthor = {\nombreAlumno\ (\email)},
	pdftitle = {\tituloProyecto},
	pdfsubject = {\nombreProyecto},
	pdfkeywords = {\keywords},
	pdfcreator = {Overleaf},
	pdfproducer = {pdflatex}
}

\hypersetup{
	colorlinks=true,
	linkcolor=black,
	urlcolor=cyan
}


\pagestyle{fancy}
\fancyhf{}
\fancyhead[LO]{\leftmark}
\fancyhead[RE]{\rightmark}
\fancyhead[RO,LE]{\textbf{\thepage}}
\setlength{\headheight}{1.5\headheight}

\renewcommand{\footrulewidth}{0pt}
\setlength{\headheight}{13.6pt}

% Fuent epor defecto
\renewcommand{\familydefault}{\sfdefault}

